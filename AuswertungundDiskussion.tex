\section{Auswertung}
\subsection{Teil 1}

Für die Länge $L$, also den Abstand zwischen Prisma und Schirm, wurde ein Wert von $385\,\mathrm{mm} \pm 2\,\mathrm{mm}$ bestimmt. 
Dieser Fehler entsteht zum einen durch die Ungenauigkeit des Lineals, zum anderen dadurch, dass das Prisma mit Augenmaß in der Mitte des Prismaständers platziert werden musste. 
Die Messung der Farbstreifen, die durch das große Prisma (im Folgenden \textit{Prisma 1} genannt) entstanden, ergab folgende Werte.

\begin{table}[h!]
\large
\centering
\label{tab:ringradien}
\begin{tabular}{|c|c|}
\hline
Farbe &Entfernung \textit{x} zum Nullpunkt in mm\\
\hline
Rot & 406 $\pm$ 1  \\
\hline
Gelb & 409 $\pm$ 1 \\
\hline  
Grün & 415 $\pm$ 2\\
\hline  
Blau & 421 $\pm$ 2\\

\hline
\end{tabular}

\caption{\textit{Die Entfernung \textit{x} zum Nullpunkt der verschiedenen Farben von Prisma 1}}



\end{table}
\noindent
Mit Hilfe Der Formel für Den Winkel $\delta_{min}$:
\begin{equation}
    \text{tan}(\delta_{min}) = \frac{x}{L} \Leftrightarrow \delta_{min} = \text{arctan}\left(\frac{x}{L}\right)
    \label{eq:5}
\end{equation}
ergibt sich folgende Tabelle für den Winkel $\delta_{min}$:

\begin{table}[h!]
\centering
\large
\label{tab:deltas}
\begin{tabular}{|c|c|c|}
\hline
\textbf{Farbe} & $\boldsymbol{\delta_{\min}}$ & $\boldsymbol{\Delta \delta_{\min}}$ \\
\hline
Rot  & $46.5^\circ$ & $ \pm 0.22^\circ$ \\
\hline
Gelb & $46.7^\circ$ & $ \pm 0.22^\circ$ \\
\hline
Grün & $47.15^\circ$ & $ \pm 0.29^\circ$ \\
\hline
Blau & $47.56^\circ$ & $ \pm 0.28^\circ$ \\
\hline
\end{tabular}
\caption{\textit{Minimaler Ablenkwinkel $\delta_{min}$ für verschiedene Farben.}}
\end{table}

\noindent
Der Fehler $\Delta \delta_{min}$ is gegeben durch:

\begin{equation}
\Delta \delta_{\min} 
= \pm \left( 
\left| \frac{\partial \delta_{\min}}{\partial x} \right| \cdot \Delta x 
+ 
\left| \frac{\partial \delta_{\min}}{\partial L} \right| \cdot \Delta L 
\right) = \pm \left(
\frac{L \cdot \Delta x}{x^2 + L^2}
+ 
\frac{x \cdot \Delta L}{x^2 + L^2}
\right)
\label{eq:delta_min_final}
\end{equation}
\noindent
Analog ie Messwerte des zweiten Prismas:

\begin{table}[h!]
\large
\centering
\label{tab:ringradien2}
\begin{tabular}{|c|c|}
\hline
Farbe &Entfernung \textit{x} zum Nullpunkt in mm\\
\hline
Rot & 530 $\pm$ 1  \\
\hline
Gelb & 543 $\pm$ 3 \\
\hline  
Grün & 553 $\pm$ 1\\
\hline  
Blau & 575 $\pm$ 3\\

\hline
\end{tabular}

\caption{\textit{Die Entfernung \textit{x} zum Nullpunkt der verschiedenen Farben von Prisma 2}}

\end{table}

\begin{table}[H]
\centering
\large
\label{tab:deltas2}
\begin{tabular}{|c|c|c|}
\hline
\textbf{Farbe} & $\boldsymbol{\delta_{\min}}$ & $\boldsymbol{\Delta \delta_{\min}}$ \\
\hline
Rot  & $46.5^\circ$ & $ \pm 0.19^\circ$ \\
\hline
Gelb & $46.7^\circ$ & $ \pm 0.28^\circ$ \\
\hline
Grün & $47.15^\circ$ & $ \pm 0.19^\circ$ \\
\hline
Blau & $57.56^\circ$ & $ \pm 0.28^\circ$ \\
\hline
\end{tabular}
\caption{\textit{Minimaler Ablenkwinkel $\delta_{min}$ für verschiedene Farben.}}
\end{table}
\noindent
Unter Verwendung von Gleichung (\ref{eq:4}) können die Brechungsindizes berechnet werden. Dabei ist $\epsilon = 60° \pm 2 °$ da es sich 
bei allen Prismen um ein gleichseiige Prismen handelt, bei welchen alle relevanten Winkel $60°$ sind. Der Fehler $\Delta \epsilon$ ist 
mit $\pm 2$ recht hoch, was daher kommt, dass die Prismen
zum Teil stark abgenutzt sind. Der Fehler in $n$ ist:
\begin{equation}
    \Delta n = \pm \left( 
    \left| \frac{\partial n}{\partial \delta_{\min}} \right| \cdot \Delta \delta_{\min}
    + 
    \left| \frac{\partial n}{\partial \varepsilon} \right| \cdot \Delta \varepsilon
    \right)
\end{equation}

\noindent
Daraus folgt:

\begin{equation}
\Delta n = \pm \left(
    \frac{
        \cos\!\left(\frac{\delta_{\min} + \varepsilon}{2}\right)
    }{
        2 \sin\!\left(\frac{\varepsilon}{2}\right)
    } \cdot \Delta \delta_{\min}
    +
    \frac{
        \cos\!\left(\frac{\varepsilon}{2}\right)
        \sin\!\left(\frac{\delta_{\min} + \varepsilon}{2}\right)
        + \sin\!\left(\frac{\varepsilon}{2}\right)
        \cos\!\left(\frac{\delta_{\min} + \varepsilon}{2}\right)
    }{
        2 \sin^2\!\left(\frac{\varepsilon}{2}\right)
    } \cdot \Delta \varepsilon
\right)
\end{equation}



\begin{table}[H]
\centering
\large
\label{tab:ns}
\begin{tabular}{|c|c|c|}
\hline
\textbf{Farbe} & $\boldsymbol{n}$ & $\boldsymbol{\Delta n}$ \\
\hline
Rot  & $1,602$ & $ \pm 0,071$ \\
\hline
Gelb & $1,605$ & $ \pm 0.071$ \\
\hline
Grün & $1,609$ & $ \pm 0,071$ \\
\hline
Blau & $1,613$ & $\pm 0,071$ \\
\hline
\end{tabular}
\caption{\textit{Brechungsindixes mit Fehlern für Prisma 1 (Kronglas)}}
\end{table}
\noindent
Da dieses Prisma einen geringeren Brechungsindex als Prisma 2 hat, handelt es sich hierbei um das Kronglas.
Die ermittelten Brechungsindizes stimmen jedoch nicht mit Literaturwerten überein, wie sie zum Beispiel in \cite{1} zu finden sind, wo ein Brechungsindex von 1,50 bis 1,54 für Silikatkronglas angegeben wird. 
Diese Werte liegen auch nicht annähernd innerhalb der Fehlergrenze $\Delta n$. Somit muss es sich um einen großen systematischen Fehler handeln. 
\noindent
Es wurde in Erwägung gezogen, dass der Nullpunkt der Skala nicht von den gemessenen $x$-Werten abgezogen wurde. 
Dies konnte jedoch anhand von Abbildung 2 ausgeschlossen werden, da die Spektren fast am Rand, also ca. bei $700\,\mathrm{mm}$, gemessen wurden. 
\noindent
Eine weitere Fehlerquelle könnte ein falscher $L$-Wert sein. Es könnte zum Beispiel sein, dass der notierte Wert von $385\,\mathrm{mm} \pm 2\,\mathrm{mm}$ gemessen wurde und die optische Bank anschließend nach hinten verschoben wurde. 
Dies würde dann zu einem falschen $\delta_{\min}$ und somit auch zu einem falschen $n$ führen. Dieses $\Delta L$ müsste aber im Bereich von etwa $200\,\mathrm{mm}$ liegen. 
\noindent
Ein weiterer Grund für diese Abweichung könnte ein Fehler am Prisma sein. Dies könnte dazu führen, dass der Strahlengang nicht wie in Abbildung 1 verläuft, was ebenfalls zu einem verfälschten $\delta_{\min}$ führen würde. 
Dies ist auch plausibel, da die verwendeten Prismen bereits starke Gebrauchsspuren aufgewiesen haben. Da sich diese Differenz aber durch alle Messungen mit verschiedenen Prismen durchzieht, ist 



\begin{table}[H]
\centering
\large
\label{tab:ns2}
\begin{tabular}{|c|c|c|}
\hline
\textbf{Farbe} & $\boldsymbol{n}$ & $\boldsymbol{\Delta n}$ \\
\hline
Rot  & $1,677$ & $ \pm 0,071$ \\
\hline
Gelb & $1,684$ & $ \pm 0.071$ \\
\hline
Grün & $1,688$ & $ \pm 0,071$ \\
\hline
Blau & $1,698$ & $\pm 0,071$ \\
\hline
\end{tabular}
\caption{\textit{Brechungsindixes mit Fehlern für Prisma 2 (Flintglas)}}
\end{table}
\noindent
Bei Prisma 2 handelt es sich um Flintglas. Dieses hat einen höheren Brechungsindex als Kronglas. Die gefundenen Brechungsindizes sind 
durchaus in dem Bereich
von üblichem Flintglas. So wird in \parencite{1} angegeben, dass Silikatflintglas bei kurzwelligen Licht einen Brechungsindex von ca. $n=1,67$ 
hat. Bei langwelligem Licht ca. $1,60$. Auch wenn diese Werte nicht
in der Fehlerschranke enthalten sind, sind die gemessen durchaus denkbar.


\subsection{Teil 2}


Mit den gleichen Formeln wie bei Teil 1 ergeben sich für die Hohlprismen, gefüllt mit destilliertem Wasser, folgende Brechungszahlen:

\begin{table}[H]
\centering
\large
\label{tab:ns3}
\begin{tabular}{|c|c|c|}
\hline
\textbf{Farbe} & $\boldsymbol{n}$ & $\boldsymbol{\Delta n}$ \\
\hline
Rot  & $1,396$ & $ \pm 0,069$ \\
\hline
Gelb & $1,397$ & $ \pm 0,069$ \\
\hline
Grün & $1,400$ & $ \pm 0,069$ \\
\hline
Blau & $1,407$ & $\pm 0,070$ \\
\hline
\end{tabular}
\caption{\textit{Brechungsindixes mit Fehlern für das Hohlprisma dass mit destilliertem Wasser gefüllt ist}}
\end{table}
\noindent
Die Literaturwerte aus \cite{Optical_Constant_of_Water} sind hier 1,340 für Blaues Licht sowie 1,335 für rotes Licht. Auch hier 
Stimmen die gemmessen Werte nicht mit den Literaturwerten überein, auch mit der Abweichungen durch Messungenaugkeiten $\Delta n$. Genau
 wie bei Prisma 1 sind
die gemmessen Brechungsindizes zu hoch. Es kann also von einem der in Teil 1 erläuterten Systemischen Fehlern ausgegangen werden.
\\
Die errechneten Brechungsindizes für Zimtsäureethylester:


\begin{table}[H]
\centering
\large
\label{tab:ns4}
\begin{tabular}{|c|c|c|}
\hline
\textbf{Farbe} & $\boldsymbol{n}$ & $\boldsymbol{\Delta n}$ \\
\hline
Rot  & $1,594$ & $ \pm 0,071$ \\
\hline
Gelb & $1,603$ & $ \pm 0,071$ \\
\hline
Grün & $1,609$ & $ \pm 0,071$ \\
\hline
Blau & $1,628$ & $\pm 0,072$ \\
\hline
\end{tabular}
\caption{\textit{Brechungsindixes mit Fehlern für das Hohlprisma dass mit Zimtsäureethylester gefüllt ist}}
\end{table}
\noindent
Hier sind Literaturwerte aus \cite{Zimt}: 1,59 Für Blaues Licht sowie 1,54 Für Rotes Licht. Auch hier sind die gemessenen Brechungsindizes zu hoch. Für eine Fehleranlyse sie Teil 1. 

\subsection{Teil 3}
\begin{figure}[H]
\centering
\includegraphics{Bilder/ngegenlampda.png}
\caption{Die Brechungsindizes der verschiedenen Prismen, aufgetragen gegen die jeweiligen Wellenlägen. Folgende Wellenlängen wurden für die Farben verwendet:
Blau: 450nm, Grün: 540nm, Gelb: 580nm, Rot: 570nm}
\end{figure} 
\noindent
Bei allen vier Materialien nimmt der Brechungsindex mit steigender Wellenlänge des 
einfallenden Lichts ab. Somit ist bei allen Materialien bei sichtbarem Licht eine normale Dispersion vorzufinden.
Wie in der Theorie besprochen, gilt somit, dass die Kreisfrequenz des einfallenden Lichts $\omega$ entweder viel größer oder viel kleiner als die Kreisfrequenz $\omega_0$ der Elektronen im Material ist.
Den größten Brechungsindex der getesteten Materialien hat Flintglas, den niedrigsten hat Wasser.
