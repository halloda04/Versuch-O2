\section{Auswertung}
\subsection{Teil 1}

Für die Länge $L$ also den Abstand zwischen Prisma und Schrim wurde ein Wert von $385$mm $\pm2 $mm. Dieser Fehler ensteht zum einen aus der
Ungenauigkeit des Lineals, zum anderen daraus das dass Prisma mit Augenmaß auf der Mitte des Prismastandes platziert werden musste. Die Messung 
der Farbstreifen welche durch das Große Prisma, im folgendem Prisma 1 gennant, ergab Folgende Werte

\begin{table}[h!]
\large
\centering
\label{tab:ringradien}
\begin{tabular}{|c|c|}
\hline
Farbe &Entfernung \textit{x} zum Nullpunkt in mm\\
\hline
Rot & 406 $\pm$ 1  \\
\hline
Gelb & 409 $\pm$ 1 \\
\hline  
Grün & 415 $\pm$ 2\\
\hline  
Blau & 421 $\pm$ 2\\

\hline
\end{tabular}

\caption{\textit{Die Entfernung \textit{x} zum Nullpunkt der verschiedenen Farben von Prisma 1}}



\end{table}
\noindent
Mit Hilfe Der Formel für Den Winkel $\delta_{min}$:
\begin{equation}
    \text{tan}(\delta_{min}) = \frac{x}{L} \Leftrightarrow \delta_{min} = \text{arctan}\left(\frac{x}{L}\right)
    \label{eq:5}
\end{equation}
ergibt sich Folgende Tabelle für den Winkel $\delta_{min}$:

\begin{table}[h!]
\centering
\large
\label{tab:deltas}
\begin{tabular}{|c|c|c|}
\hline
\textbf{Farbe} & $\boldsymbol{\delta_{\min}}$ & $\boldsymbol{\Delta \delta_{\min}}$ \\
\hline
Rot  & $46.5^\circ$ & $ \pm 0.22^\circ$ \\
\hline
Gelb & $46.7^\circ$ & $ \pm 0.22^\circ$ \\
\hline
Grün & $47.15^\circ$ & $ \pm 0.29^\circ$ \\
\hline
Blau & $47.56^\circ$ & $ \pm 0.28^\circ$ \\
\hline
\end{tabular}
\caption{\textit{Minimaler Ablenkwinkel $\delta_{min}$ für verschiedene Farben.}}
\end{table}


Der Fehler $\Delta \delta_{min}$ is gegeben durch:

\begin{equation}
\Delta \delta_{\min} 
= \pm \left( 
\left| \frac{\partial \delta_{\min}}{\partial x} \right| \cdot \Delta x 
+ 
\left| \frac{\partial \delta_{\min}}{\partial L} \right| \cdot \Delta L 
\right) = \pm \left(
\frac{L \cdot \Delta x}{x^2 + L^2}
+ 
\frac{x \cdot \Delta L}{x^2 + L^2}
\right)
\label{eq:delta_min_final}
\end{equation}

Die Messwerte des Zweiten Prismas:

\begin{table}[h!]
\large
\centering
\label{tab:ringradien2}
\begin{tabular}{|c|c|}
\hline
Farbe &Entfernung \textit{x} zum Nullpunkt in mm\\
\hline
Rot & 530 $\pm$ 1  \\
\hline
Gelb & 543 $\pm$ 3 \\
\hline  
Grün & 553 $\pm$ 1\\
\hline  
Blau & 575 $\pm$ 3\\

\hline
\end{tabular}

\caption{\textit{Die Entfernung \textit{x} zum Nullpunkt der verschiedenen Farben von Prisma 2}}

\end{table}

\begin{table}[H]
\centering
\large
\label{tab:deltas2}
\begin{tabular}{|c|c|c|}
\hline
\textbf{Farbe} & $\boldsymbol{\delta_{\min}}$ & $\boldsymbol{\Delta \delta_{\min}}$ \\
\hline
Rot  & $46.5^\circ$ & $ \pm 0.19^\circ$ \\
\hline
Gelb & $46.7^\circ$ & $ \pm 0.28^\circ$ \\
\hline
Grün & $47.15^\circ$ & $ \pm 0.19^\circ$ \\
\hline
Blau & $57.56^\circ$ & $ \pm 0.28^\circ$ \\
\hline
\end{tabular}
\caption{\textit{Minimaler Ablenkwinkel $\delta_{min}$ für verschiedene Farben.}}
\end{table}
Unter Verwendung von Gleichung (\ref{eq:4}) können die Brechungsindizes berechnet werden. Dabei ist $\epsilon = 60° \pm 2 °$ da es sich 
bei allen Prismen um ein gleichseiige Prismen handelt, bei welchen alle relevanten Winkel $60°$ sind. Der Fehler $\Delta \epsilon$ ist 
mit $\pm 2$ recht hoch, was daher kommt, dass die Prismen
zum Teil stark abgenutzt sind. Der Fehler in $n$ ist:
\begin{equation}
    \Delta n = \pm \left( 
    \left| \frac{\partial n}{\partial \delta_{\min}} \right| \cdot \Delta \delta_{\min}
    + 
    \left| \frac{\partial n}{\partial \varepsilon} \right| \cdot \Delta \varepsilon
    \right)
\end{equation}

\noindent
Daraus folgt:

\begin{equation}
\Delta n = \pm \left(
    \frac{
        \cos\!\left(\frac{\delta_{\min} + \varepsilon}{2}\right)
    }{
        2 \sin\!\left(\frac{\varepsilon}{2}\right)
    } \cdot \Delta \delta_{\min}
    +
    \frac{
        \cos\!\left(\frac{\varepsilon}{2}\right)
        \sin\!\left(\frac{\delta_{\min} + \varepsilon}{2}\right)
        + \sin\!\left(\frac{\varepsilon}{2}\right)
        \cos\!\left(\frac{\delta_{\min} + \varepsilon}{2}\right)
    }{
        2 \sin^2\!\left(\frac{\varepsilon}{2}\right)
    } \cdot \Delta \varepsilon
\right)
\end{equation}



\begin{table}[H]
\centering
\large
\label{tab:ns}
\begin{tabular}{|c|c|c|}
\hline
\textbf{Farbe} & $\boldsymbol{n}$ & $\boldsymbol{\Delta n}$ \\
\hline
Rot  & $1,602$ & $ \pm 0,071$ \\
\hline
Gelb & $1,605$ & $ \pm 0.071$ \\
\hline
Grün & $1,609$ & $ \pm 0,071$ \\
\hline
Blau & $1,613$ & $\pm 0,071$ \\
\hline
\end{tabular}
\caption{\textit{Brechungsindixes mit Fehlern für Prisma 1 (Kronglas)}}
\end{table}
Da dieses Prisma einen geringeren Brechungsindex als Prisma 2 hat, handelt es sich hierei um das Kronglas.
Die ermittelten Brechungsindizes stimmen aber nicht mit Literaturwerten übererinander wie sie
zum Beispiel in [\parencite{tipler}] zu Finden wahren. Wo ein Brechungsindex von 1,54 bis 1,50 für das Silicatkronglas angegeben ist. 
Diese Werte sind auch nicht anähernd in der Fehlergrenze $\Delta n$ enthalten. Somit muss es sich hier um einen großen Systemischen 
Fehler handeln. 
Es wurde in Erwägung gezogen, dass der Nullpunkt der Skala nicht von den gemmessen $x$ abgezogen wurde.
Dies wurde aber anhand von Abbildung 2 als nicht möglich bestätigt da die Spektren fast am Rand also ca. bei $700$mm gemmesen wurden.
Eine Weitere Fehlerquelle könnte ein Falscher $L$ Wert sein. Es könnte zum Beispiel sein, das der notierte Wert von $385$mm $\pm2$mm 
gemmessen wurde und die optische Bank anschließend nach hinten verschoben wurde.
Dies würde dann zu einem Falschen $\delta_{min}$ und somit auch zu einem Falschen $n$ führen. Dieses $\Delta L$ müsste im Bereich von 
$200$mm sein. Ein Weiterer Grund für diese Abweichung könnte ein Fehler am Prisma 
sein. Das könnte dazu führen, dass der Strahlengang nicht so wie in Abbilung (1) ist. Das würde dann ebenfalls zu einem verfälschten 
$\delta_{min}$ führen. Dies kann vorallem in Betracht gezogen werden, da die Verwendeten
Prismen schon sehr starke gebrauchspruren Gezeit haben.


\begin{table}[H]
\centering
\large
\label{tab:ns2}
\begin{tabular}{|c|c|c|}
\hline
\textbf{Farbe} & $\boldsymbol{n}$ & $\boldsymbol{\Delta n}$ \\
\hline
Rot  & $1,677$ & $ \pm 0,071$ \\
\hline
Gelb & $1,684$ & $ \pm 0.071$ \\
\hline
Grün & $1,688$ & $ \pm 0,071$ \\
\hline
Blau & $1,698$ & $\pm 0,071$ \\
\hline
\end{tabular}
\caption{\textit{Brechungsindixes mit Fehlern für Prisma 2 (Flintglas)}}
\end{table}
\noindent
Bei Prisma 2 handelt es sich um Flintglas. Dieses hat einen höhren Brechungsindex als Kronglas. Die gefundenen Brechungsindizes sind hier 
durchaus in dem Bereich
von üblichen Flintglas. So wird in [\parencite{tipler}] angegeben, dass Silicatflinglas bei kurzwelligenlicht einen Brechungsindex von ca. $n=1,67$ 
hat. Bei Langweligem Licht ca. $1,6$. Auch wenn diese Werte nicht
in der Fehlerschranke enthalten sind, sind die Gemmessenen Brechungsindizes im Bereich von typischem Flintglas

\newpage
\subsection{Teil 2}


Mit den gleichen Formel wie bei Teil 1, ergeben sich für die Holhmprismen gefüllt mit destilliertem Wasser und 
Zimtsäureethylester folgende Brechungszahlen 
Für Das Destillierte Wasser:

\begin{table}[H]
\centering
\large
\label{tab:ns3}
\begin{tabular}{|c|c|c|}
\hline
\textbf{Farbe} & $\boldsymbol{n}$ & $\boldsymbol{\Delta n}$ \\
\hline
Rot  & $1,396$ & $ \pm 0,069$ \\
\hline
Gelb & $1,397$ & $ \pm 0,069$ \\
\hline
Grün & $1,400$ & $ \pm 0,069$ \\
\hline
Blau & $1,407$ & $\pm 0,070$ \\
\hline
\end{tabular}
\caption{\textit{Brechungsindixes mit Fehlern für das Hohlprisma dass mit destilliertem Wasser gefüllt ist}}
\end{table}
\noindent
Die Literaturwerte aus [\parencite{Optical_Constant_of_Water}] sind hier 1,340 für Blaues Licht sowie 1,335 für rotes Licht. Auch hier 
Stimmen die gemmessen Werte nicht mit den Literaturwerten überein, auch mit der Abweichungen durch Messungenaugkeiten $\Delta n$. Genau
 wie bei Prisma 1 sind
die gemmessen Brechungsindizes zu hoch. Es kann also von einem der in Teil 1 erläuterten Systemischen Fehlern ausgegangen werden.

Für die Zimtsäureethylester:

\begin{table}[H]
\centering
\large
\label{tab:ns4}
\begin{tabular}{|c|c|c|}
\hline
\textbf{Farbe} & $\boldsymbol{n}$ & $\boldsymbol{\Delta n}$ \\
\hline
Rot  & $1,594$ & $ \pm 0,071$ \\
\hline
Gelb & $1,603$ & $ \pm 0,071$ \\
\hline
Grün & $1,609$ & $ \pm 0,071$ \\
\hline
Blau & $1,628$ & $\pm 0,072$ \\
\hline
\end{tabular}
\caption{\textit{Brechungsindixes mit Fehlern für das Hohlprisma dass mit Zimtsäureethylester gefüllt ist}}
\end{table}
\newpage
\subsection{Teil 3}
\begin{figure}[H]
\centering
\includegraphics{Bilder/ngegenlampda.png}
\caption{Die Brechungsindezes $n$ der verschiedenen Prismen aufgetragen gegen die Wellenlänge $\lambda$. Es wurden Folgende Wellenlängen 
genommen: Blau: $450$nm, Grün: $530$nm, Gelb: $580$nm, Rot: $670$nm}
\end{figure}    

