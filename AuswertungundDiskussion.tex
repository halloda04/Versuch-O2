\section{Auswertung}
\subsection{Teil 1}

Für die Länge $L$ also den Abstand zwischen Prisma und Schrim wurde ein Wert von $385$mm $\pm2 $mm. Dieser Fehler ensteht zum einen aus der
Ungenauigkeit des Lineals, zum anderen daraus das dass Prisma mit Augenmaß auf der Mitte des Prismastandes platziert werden musste. Die Messung 
der Farbstreifen welche durch das Große Prisma, im folgendem Prisma 1 gennant, ergab Folgende Werte

\begin{table}[h!]
\large
\centering
\label{tab:ringradien}
\begin{tabular}{|c|c|}
\hline
Farbe &Entfernung \textit{x} zum Nullpunkt in mm\\
\hline
Rot & 406 $\pm$ 1  \\
\hline
Gelb & 409 $\pm$ 1 \\
\hline  
Grün & 415 $\pm$ 2\\
\hline  
Blau & 421 $\pm$ 2\\

\hline
\end{tabular}

\caption{\textit{Die Entfernung \textit{x} zum Nullpunkt der verschiedenen Farben von Prisma 1}}



\end{table}
\noindent
Mit Hilfe Der Formel für Den Winkel $\delta_{min}$:
\begin{equation}
    \text{tan}(\delta_{min}) = \frac{x}{L} \Leftrightarrow \delta_{min} = \text{arctan}\left(\frac{x}{L}\right)
    \label{eq:5}
\end{equation}
ergibt sich Folgende Tabelle für den Winkel $\delta_{min}$:

\begin{table}[h!]
\centering
\large
\label{tab:deltas}
\begin{tabular}{|c|c|c|}
\hline
\textbf{Farbe} & $\boldsymbol{\delta_{\min}}$ & $\boldsymbol{\Delta \delta_{\min}}$ \\
\hline
Rot  & $46.5^\circ$ & $ \pm 0.22^\circ$ \\
\hline
Gelb & $46.7^\circ$ & $ \pm 0.22^\circ$ \\
\hline
Grün & $47.15^\circ$ & $ \pm 0.29^\circ$ \\
\hline
Blau & $47.56^\circ$ & $ \pm 0.28^\circ$ \\
\hline
\end{tabular}
\caption{\textit{Minimaler Ablenkwinkel $\delta_{min}$ für verschiedene Farben.}}
\end{table}


Der Fehler $\Delta \delta_{min}$ is gegeben durch:

\begin{equation}
\Delta \delta_{\min} 
= \pm \left( 
\left| \frac{\partial \delta_{\min}}{\partial x} \right| \cdot \Delta x 
+ 
\left| \frac{\partial \delta_{\min}}{\partial L} \right| \cdot \Delta L 
\right) = \pm \left(
\frac{L \cdot \Delta x}{x^2 + L^2}
+ 
\frac{x \cdot \Delta L}{x^2 + L^2}
\right)
\label{eq:delta_min_final}
\end{equation}

Die Messwerte des Zweiten Prismas:

\begin{table}[h!]
\large
\centering
\label{tab:ringradien2}
\begin{tabular}{|c|c|}
\hline
Farbe &Entfernung \textit{x} zum Nullpunkt in mm\\
\hline
Rot & 530 $\pm$ 1  \\
\hline
Gelb & 543 $\pm$ 3 \\
\hline  
Grün & 553 $\pm$ 1\\
\hline  
Blau & 575 $\pm$ 3\\

\hline
\end{tabular}

\caption{\textit{Die Entfernung \textit{x} zum Nullpunkt der verschiedenen Farben von Prisma 2}}

\end{table}

\begin{table}[H]
\centering
\large
\label{tab:deltas2}
\begin{tabular}{|c|c|c|}
\hline
\textbf{Farbe} & $\boldsymbol{\delta_{\min}}$ & $\boldsymbol{\Delta \delta_{\min}}$ \\
\hline
Rot  & $46.5^\circ$ & $ \pm 0.19^\circ$ \\
\hline
Gelb & $46.7^\circ$ & $ \pm 0.28^\circ$ \\
\hline
Grün & $47.15^\circ$ & $ \pm 0.19^\circ$ \\
\hline
Blau & $57.56^\circ$ & $ \pm 0.28^\circ$ \\
\hline
\end{tabular}
\caption{\textit{Minimaler Ablenkwinkel $\delta_{min}$ für verschiedene Farben.}}
\end{table}
Unter Verwendung von Gleichung (\ref{eq:4}) können die Brechungsindizes berechnet werden. Dabei ist $\epsilon = 60° \pm 2 °$ da es sich bei allen Prismen um ein gleichseiige Prismen handelt, bei welchen alle relevanten Winkel $60°$ sind. Der Fehler $\Delta \epsilon$ ist mit $\pm 2$ recht hoch, was daher kommt, dass die Prismen
zum Teil stark abgenutzt sind. Der Fehler in $n$ ist:
\begin{equation}
    \Delta n = \pm \left( 
    \left| \frac{\partial n}{\partial \delta_{\min}} \right| \cdot \Delta \delta_{\min}
    + 
    \left| \frac{\partial n}{\partial \varepsilon} \right| \cdot \Delta \varepsilon
    \right)
\end{equation}

Daraus folgt:

\begin{equation}
\Delta n = \pm \left(
    \frac{
        \cos^2\!\left(\frac{\delta_{\min} + \varepsilon}{2}\right)
    }{
        2 \sin\!\left(\frac{\varepsilon}{2}\right)
    } \cdot \Delta \delta_{\min}
    +
    \frac{
        \cos\!\left(\frac{\varepsilon}{2}\right)
        \sin\!\left(\frac{\delta_{\min} + \varepsilon}{2}\right)
        + \sin\!\left(\frac{\varepsilon}{2}\right)
        \cos\!\left(\frac{\delta_{\min} + \varepsilon}{2}\right)
    }{
        2 \sin^2\!\left(\frac{\varepsilon}{2}\right)
    } \cdot \Delta \varepsilon
\right)
\end{equation}

