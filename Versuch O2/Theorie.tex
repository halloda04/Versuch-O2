\section{Theoretische Grundlagen}

%\begin{wrapfigure}{r}{0.4\textwidth} % r=rechts, l=links, 0.4=Breite 
%        \centering
%        \includegraphics[width=0.9\linewidth]{Bilder/Titelbild.png}
%        \caption{Veranschaulichung der LSA}
%       \label{fig:SpASkizze}
%\end{wrapfigure}

\subsection{Lichtbrechung}
Das für uns sichtbare Licht besteht aus vielen Wellenlängen an
elektromagnetischen Wellen. Damit diese verschiedenen Wellenlängen
einzeln betrachtet werden können, kann man das sichtbare Licht
mithilfe eines optischen Prismas brechen. Dieser Vorgang kann in
Abbildung 1 beobachtet werden. Die verschiedenen Brechungswinkel
hängen mit den unterschiedlich starken Brechungen der
unterschiedlichen elektromagnetischen Wellen zusammen.
Diese Brechung ist als Brechungsindex bekannt. Der Zusammenhang
zwischen dem Brechungsindex und der Wellenlänge der
elektromagnetischen Welle wird zu einem späteren Zeitpunkt noch
genauer betrachtet.
\begin{figure}[!htb]
\centering
\includegraphics{Bilder/Strahlengang O2.png}
\caption{Abbildug 1: Der Strahlengang eines gebrochenen Lichtstrahles innerhalb eines Prisma mit den wichtige Winkeln}
\end{figure}

Um den Brechungsindex für einzelne Wellenlängen bestimmen zu
können, betrachtet man zunächst den Ablenkungswinkel $\delta$.
Dieser wird minimal für $\alpha_1 = \alpha_2$ und $\beta_1 = \beta_2$, womit für beide Winkel
\begin{equation}
    \beta_1 = \frac{\varepsilon}{2}
    \label{eq:1}
\end{equation}

und

\begin{equation}
    \alpha_1 = \frac{\delta_{\text{min}} + \varepsilon}{2}
    \label{eq:2}
\end{equation}
gilt. Wenn diese gleichungen nun mit dem snelliusschen Brechungsgesetz kombiiert werde

\begin{equation}
    n_1 \cdot \sin(\alpha_1) = n_2 \cdot \sin(\beta_1)
    \label{eq:3}
\end{equation}

mit $n_1 = n_{\text{Luft}} \stackrel{!}{=} 1$ und den Winkeln aus Gleichung (\ref{eq:1}) und (\ref{eq:2}) folgt für den Brechungsindex $n$ des Prismas

\begin{equation}
    n = n_2 = \frac{\sin\left(\frac{\delta_{\text{min}} + \varepsilon}{2}\right)}{\sin\left(\frac{\varepsilon}{2}\right)} .
    \label{eq:4}
\end{equation}


\subsection{Dispersion}
Dispersion beschreibt in dem betrachteten Fall den Zusammenhang
zwischen der Wellenlänge der betrachteten elektromagnetischen
Wellen und dem Brechungsindex. Diese Abhängigkeit lässt sich am
Thomson-Atommodell erklären. Dieses Atommodell beschreibt ein Atom
als homogen positiv geladene Kugel, in der die Elektronen frei
beweglich sind. In diesem Modell können Elektronen durch
elektromagnetische Wellen zu Schwingungen angeregt werden.
Diese Schwingung lässt sich durch die folgende Differentialgleichung
beschreiben.
\begin{equation}
    m_0r*\ddot{r} + m\gamma*\dot{r} + m_0*\omega_0^2*r=-e*E
\end{equation}
Wobei $\omega_0$ die Eigenfrequenz der Elektronen beschreibt,
$m_0 * \gamma * \dot{r}$ ist die Darstellung des Dämpfungsterms, $e$ die
Elementarladung und $E$ die elektrische Feldstärke des anregenden
Photons.\\
Bei der Dispersion können drei Fälle auftreten. Der erste Fall
ist für $\omega \ll \omega_0$, bei diesem Fall kann man die Dämpfung
vernachlässigen, und es kommt zu normaler Dispersion. Bei
$\omega \approx \omega_0$ kommt es zu anomaler Dispersion. Der
letzte Fall ist bei $\omega_0 \ll \omega$, dieses Ereignis ist
als Resonanzkatastrophe bekannt und tritt zum Beispiel auf, wenn
eine Armee im Gleichschritt über eine große Brücke marschiert.
Bei diesem Experiment sind aber nur die normale und anomale
Dispersion wichtig. Bei normaler Dispersion nimmt der
Brechungsindex mit steigender Frequenz/sinkender Wellenlänge zu,
während bei der anomalen Dispersion genau das Gegenteil geschieht,
dort nimmt der Brechungsindex bei steigender Wellenlänge zu.