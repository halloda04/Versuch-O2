\section{Versuchsbeschreibung}
\subsection{Versuchsaufbau}
Der Versuchsaufbau besteht aus einer Lampe, die hinter einer verstellbaren
Spaltblende montiert wird. Vor der Spaltblende wird eine Halterung für
Farbfilter und eine Sammellinse montiert. Als letztes Bauteil auf der
optischen Bank wird ein Podest benötigt, auf welches während der
Versuchsdurchführung Prismen gestellt werden können. Vor dem Podest wird
nun ein Schirm mit einer montierten Skala aufgebaut. Das Podest hat eine
Entfernung von 38,5 cm zum Schirm.
\begin{figure}[!htb]
\centering
\includegraphics[scale=0.08]{Bilder/Richtiger Aufbau.jpg}
\caption{Ein Bild vom Versuchsaufbau, von vorne nach hinten. Lampe, Spaltblende, Farbfilterhalterung, Sammellinse, Podest und Schirm.}
\end{figure}


\subsection{Versuchsdurchführung}
Nun zur Versuchsdurchführung: Zuerst wird die Lampe eingeschaltet und
mithilfe der Spaltblende und der Sammellinse auf dem Schirm fokussiert.
Nun wird der fokussierte Punkt als $X_0$ vermerkt und die Entfernung
des Podests zum Schirm gemessen. Sobald diese Schritte erledigt sind,
kann mit der Durchführung richtig begonnen werden.
Für den ersten Teil der Messung werden die zwei Vollprismen verwendet.
Dazu wird das erste Prisma auf das Podest in den Lichtstrahl gestellt
und so lange vorsichtig gedreht, bis das entstehende Farbspektrum nicht
mehr weiter in Richtung des Punktes $X_0$ wandert. Nun werden
nacheinander die Farbfilter in die Farbfilterhalterung gesteckt, um die
genaue Messung der verschiedenen Farben zu vereinfachen, denn jetzt wird
die Entfernung der jeweiligen Farblinie zu dem Punkt $X_0$ ermittelt.
Sobald dies für alle Farbfilter durchgeführt wurde, werden die beiden
Prismen getauscht und der gesamte Vorgang wird wiederholt.
Der zweite Teil des Experiments läuft analog zum ersten Teil ab, doch
jetzt werden nicht die Vollprismen verwendet, sondern die Hohlprismen,
die mit Flüssigkeit gefüllt sind.