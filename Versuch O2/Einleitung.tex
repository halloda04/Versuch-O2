\section{Einleitung}

Überall auf dem Globus ist die Brechung von Licht ein wichtiger 
Bestandteil des alltäglichen Lebens. Der wahrscheinlich wichtigste 
Anwendungsfall ist die Brille, beim Anwendungsfall der Brille kommt es 
aber nicht nur zu einfacher Brechung von Licht sondern auch zu 
Dispersion. Brillen sind aber nicht die einzigen Anwendungspunkte bei 
denen die Eigenschaften von Linsen und Prismen genutzt werden. Ein paar 
weitere Anwendungen sind zum Beispiel: Mikroskope, Spiegelreflex Kameras 
oder Fernrohre. Bei all diesen Fällen kommt es auch zur Dispersion von 
Licht. Die Dispersion von Licht beschreibt die Abhängigkeit des 
Brechungsindex von der Wellenlänge. Bei extremen Fällen in der 
Anwendung kann es somit vorkommen dass das Licht welches wir als 
weiß warnehmen in seine einzelne Bestandteile aufgespalten wird und man 
somit ein stark verschwommenes Bild nur warnehmen kann oder gar bei 
messungen starke verfälschungen bekommt. Der Versuch ist der Kategorie 
der Optik zuzuordnen und wird durchgeführt um die stärke der Dispersion 
in Abhängigkeit mit dem Material Experimentell Nachzuweisen.