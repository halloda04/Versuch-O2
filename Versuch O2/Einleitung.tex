\section{Einleitung}

Überall auf dem Globus ist die Brechung von Licht ein wichtiger
Bestandteil des alltäglichen Lebens. Der wahrscheinlich wichtigste
Anwendungsfall ist die Brille. Beim Anwendungsfall der Brille kommt es
aber nicht nur zur einfachen Brechung von Licht, sondern auch zur
Dispersion. Brillen sind jedoch nicht die einzigen Anwendungspunkte, bei
denen die Eigenschaften von Linsen und Prismen genutzt werden. Ein paar
weitere Anwendungen sind zum Beispiel: Mikroskope, Spiegelreflexkameras
oder Fernrohre. Bei all diesen Fällen kommt es ebenfalls zur Dispersion
von Licht. Die Dispersion von Licht beschreibt die Abhängigkeit des
Brechungsindex von der Wellenlänge. In extremen Fällen kann es in der
Anwendung vorkommen, dass das Licht, welches wir als
weiß wahrnehmen, in seine einzelnen Bestandteile aufgespalten wird.
Dadurch kann ein stark verschwommenes Bild entstehen oder es können bei
Messungen starke Verfälschungen auftreten. Der Versuch ist der Kategorie der Optik zuzuordnen und wird
durchgeführt, um die Stärke der Dispersion in Abhängigkeit vom Material
experimentell nachzuweisen