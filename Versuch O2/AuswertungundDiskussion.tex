\section{Auswertung}
\subsection{Teil 1}

Für die Länge $L$ also den Abstand zwischen Prisma und Schrim wurde ein Wert von $385$mm $\pm2 $mm. Dieser Fehler ensteht zum einen aus der
Ungenauigkeit des Lineals, zum anderen daraus das dass Prisma mit Augenmaß auf der Mitte des Prismastandes platziert werden musste. Die Messung 
der Farbstreifen welche durch das Große Prisma, im folgendem Prisma 1 gennant, ergab Folgende Werte

\begin{table}[h!]
\large
\centering
\label{tab:ringradien}
\begin{tabular}{|c|c|}
\hline
Farbe &Entfernung \textit{x} zum Nullpunkt in mm\\
\hline
Rot & 406 $\pm$ 1  \\
\hline
Gelb & 409 $\pm$ 1 \\
\hline  
Grün & 415 $\pm$ 2\\
\hline  
Blau & 421 $\pm$ 2\\

\hline
\end{tabular}

\caption{\textit{Die Entfernung \textit{x} zum Nullpunkt der verschiedenen Farben von Prisma 1}}



\end{table}

Mit Hilfe Der Formel für Den Winkel $\delta_{min}$:

\[
    \text{tan}(\delta_{min}) = \frac{x}{L}
    \label{eq:5}
\]

